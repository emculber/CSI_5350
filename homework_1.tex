\documentclass{report}
\usepackage[showframe=false]{geometry}
\usepackage{titlesec}
\usepackage{amsmath}
\usepackage{graphicx}
\usepackage{qtree}
\usepackage{lscape}

\pagenumbering{gobble}

\geometry{tmargin=60pt,bmargin=90pt,lmargin=90pt,
rmargin=90pt}

\titleformat{\chapter}{\normalfont\huge}{\thechapter.}{20pt}{\huge}
\titlespacing*{\chapter} {0pt}{0pt}{10pt}

\begin{document}

\chapter{Grammer}

\section{Root}

\begin{equation}
\begin{split}
  <root> & ::= <stmt-list> \\
  <stmt-list> & ::= <stmt> <stmt-list> | <stmt> \\
  <stmt> & ::= <assignment> | <expr> | <if-stmt> | <loopexp> \\
  &| <function-declaration> | <function-invocation> \\
\end{split}
\end{equation}

\section{Assignment}

\begin{equation}
\begin{split}
  <assignment> & ::= <declarator>; | <reassign>; \\
\end{split}
\end{equation}

\subsection{Declarator}

\begin{equation}
\begin{split}
  <declarator> & ::= <type-name>  <declarator-list> \\
  <type-name> & ::= boolean  |  int  |  float  |  char  |  string \\
  <declarator-list> & ::= <delarator> | <declarator>, <declarator-list> \\
  <declarator> & ::= id  | id = <expr> | id<array-list> | id<array-list> = \{ <list-values> \} \\
  <array-list> & ::= [<value>] | [<value>]<array-list> \\
  <list-values> & ::= <expr>, <list-values> | <expr> | \{ <list-values> \} \\
  <value> & ::= integer | float | <empty> \\
\end{split}
\end{equation}

\subsection{Reassignment}

\begin{equation}
\begin{split}
  <reassign> & ::= id = <expr>  | id<reassign-array-list> = <expr> \\
  <reassign-array-list> & ::= [<reassign-value>] | [<reassign-value>]<reassign-array-list> \\
  <reassign-value> & ::= integer | float \\
\end{split}
\end{equation}

\section{Expressions}

\begin{equation}
\begin{split}
  <expr> & ::= <stringexp> | <logconj> | <addsubexp> \\
\end{split}
\end{equation}

\subsection{Boolean Operations}

\begin{equation}
\begin{split}
  <logconj> & ::= <logconj> || <logdisj> | <logdisj> \\
  <logdisj> & ::= <logdisj> \&\& <logneg> | <logneg> \\
  <logneg> & ::= !<compopt> | <compopt> \\
  <compopt> & ::=  <stringexp>\  ==\  <stringexp> | <addsubexp>\  >=\  <addsubexp> \\
            &     | <addsubexp>\  <=\  <addsubexp> | <stringexp>\  !=\  <stringexp> \\
\end{split}
\end{equation}

\subsection{String operators}

\begin{equation}
\begin{split}
  <stringexp> & ::= <stringroot> + <stringexp> | <stringroot> | <addsubexp>  \\
  <stringroot> & ::= string | id \\
\end{split}
\end{equation}

\subsection{Arithmatic expressions}

\begin{equation}
\begin{split}
  <addsubexp> & ::= <addsubexp> + <muldivexp> | <addsubexp> - <muldivexp> | <muldivexp> \\
  <muldivexp> & ::= <muldivexp> * <expexp> | <muldivexp> / <expexp> | <expexp> \\
  <expexp> & ::= <rootexp> \# <expexp> | <rootexp> \\
  <rootexp> & ::= (<expr>) |  id  |  integer  |  float \\
\end{split}
\end{equation}

\section{Conditional}

\begin{equation}
\begin{split}
  <if-stmt> & ::= if (<expr>) \{ <stmt-list> \} else \{ <stmt-list> \} | if (<expr>) \{ <stmt-list> \} \\
\end{split}
\end{equation}

\section{Loops}

\begin{equation}
\begin{split}
  <loopexp> & ::= <forloop> | <whileloop> \\
  <forloop> & ::= for (assignment; <compopt>; <reassign>) \{ <stmt-list> \} \\
  <whileloop> & ::= for (<compopt>) \{ <stmt-list> \} \\
\end{split}
\end{equation}

\section{Functions}

\subsection{Declare}

\begin{equation}
\begin{split}
  <function-declaration> & ::= <func-type> id (<declare-arg-list>) \{ <stmt-list> \} \\
  <func-type> & ::= <type-name> | void \\
  <declare-arg-list> & ::= <declare-arg>, <arg-list> | <declare-arg> \\
  <declare-arg> & ::= <type-name> id | <empty> \\
\end{split}
\end{equation}

\subsection{Invocation}

\begin{equation}
\begin{split}
  <function-invocation> & ::= <type-name> id = id(arg-list); | id(arg-list); \\
  <arg-list> & ::= <arg>, <arg-list> | <arg> \\
  <arg> & ::= <declarator> | id | <empty> \\
\end{split}
\end{equation}

\end{document}
